\section{Introducción}
	Las conferencias Macy fueron una serie de reuniones en las cuales participaban especialistas de diversas disciplinas. 
	Estas reuniones se realizaban en Nueva York bajo la dirección de Frank Fremont-Smith en la Fundación Josiah Macy Jr, las cuales se llevaron a cabo durante los años de 1946 hasta 1953.

	El objetivo especifico de las mismas, era promover la comunicación entre distintas áreas científicas y restaurar la unidad entre la ciencia (debido a los sucesos acontecidos en la Segunda Guerra Mundial). La fundación mencionada anteriormente, desarrollo específicamente dos innovaciones las cuales buscaban fortalecer y facilitar los intercambios interdisciplinarios y multidisciplinarios las cuales fueron las conferencias Macy (oral) y las transacciones Macy (escritos).

	El formato de las conferencias Macy era básicamente un grupo de investigadores dialogando sobre investigaciones que se encontraban desarrollando. El objetivo explicito de las mismas era que un público extenso pudiera escuchar el intercambio de ideas entre los expertos y así poder dar su opinión sobre su trabajo.

	Los participantes eran científicos lideres de un gran rango de campos, los cuales eran elegidos debido a su capacidad para interactuar en conversaciones interdisciplinarias, o por tener conocimientos de múltiples ramas.